
% Save this as tutorial.tex for the lwarp package tutorial.

\documentclass[12pt,a4paper]{report}

\usepackage{iftex}

% --- LOAD FONT SELECTION AND ENCODING BEFORE LOADING LWARP ---

\ifPDFTeX
\usepackage{lmodern}            % pdflatex or dvi latex
\usepackage[utf8]{inputenc}
\else
\usepackage{fontspec}           % XeLaTeX or LuaLaTeX
\fi
\usepackage[magyar]{babel}

% --- LWARP IS LOADED NEXT ---
\usepackage[
% Filename of the homepage.
   HomeHTMLFilename=index,     
   HTMLFilename={node-},       % Filename prefix of other pages.
%   IndexLanguage=english,      % Language for xindy index, glossary.
%   REMARK (szgabbor): After commenting, I was able to compile
%   xindex,
%   xindy,
   %latexmk,                    % Use latexmk to compile.
%   OSWindows,                  % Force Windows. (Usually automatic.)
%   mathjax,                    % Use MathJax to display math.
]{lwarp}
% \boolfalse{FileSectionNames}  % If false, numbers the files.

% --- LOAD PDFLATEX MATH FONTS HERE ---
% PREAMBULUM IDE

\usepackage{amssymb}
\usepackage{amsmath}
\usepackage{graphicx}
\usepackage{tikz}
\usepackage{fancyvrb-ex}
\usepackage[a4paper]{geometry}

% --- OTHER PACKAGES ARE LOADED AFTER LWARP ---
%\usepackage{makeidx} \makeindex
\usepackage{xindex} \makeindex
%\usepackage{xindex} %\makeindex
%\usepackage[abbreviations,symbols,xindy]{glossaries-extra} 
\makeindex
\usepackage{xcolor}             % (Demonstration purposes only.)

\usepackage{mdframed} 
\definecolor{bgInline}{rgb}{0.9, 0.9, 0.9}
\definecolor{bgBox}{rgb}{0.8, 0.8, 0.8}

\mdfdefinestyle{box}{
    linecolor=bgBox,
	backgroundcolor=bgBox,
	hidealllines=true,
	roundcorner=10pt,
}

\usepackage{minted} % http://tug.ctan.org/macros/latex/contrib/minted/minted.pdf

% https://tex.stackexchange.com/questions/504036/using-mdframed-with-inputminted
\let\inputmintedorg\inputminted
\newcommand{\mintedframe}[3][]{%
\begin{mdframed}[style=box]
\inputminted[#1]{#2}{#3}
\end{mdframed}
}

% https://tex.stackexchange.com/questions/89873/repeating-environment-contents-twice
\newenvironment{tikzExample}
 {\VerbatimOut{\jobname.tmp}}
 {\endVerbatimOut
  \mintedframe[tabsize=4,breaklines,breaksymbolleft={}]{latex}{\jobname.tmp}
  \input{\jobname.tmp}}

\newcommand{\code}[1]{\mintinline[bgcolor=bgInline]{latex}{#1}}

% https://tex.stackexchange.com/questions/10326/how-to-set-the-chapter-style-in-report-class
\usepackage{titlesec}
\titleformat{\chapter}
  {\normalfont\LARGE\bfseries}{\thechapter.}{1em}{}
\titlespacing*{\chapter}{0pt}{3.5ex plus 1ex minus .2ex}{2.3ex plus .2ex}

\usepackage{hyperref,cleveref}  % LOAD THESE LAST!

% --- LATEX AND HTML CUSTOMIZATION ---
\title{TikZ gyorstalpaló, példák}
\author{Bertalan Dávid, Alexy Marcell (technikai megvalósítás)}
\setcounter{tocdepth}{1}        % Include subsections in the \TOC.
\setcounter{secnumdepth}{2}     % Number down to subsections.
\setcounter{FileDepth}{0}       % 1: Split \HTML\ files at sections ; 0: split at chapter
\booltrue{CombineHigherDepths}  % Combine parts/chapters/sections
\setcounter{SideTOCDepth}{1}    % Include subsections in the side\TOC
%\HTMLTitle{Webpage Title}       % Overrides \title for the web page.
%\HTMLAuthor{\theauthor}        % Sets the HTML meta author tag.
\HTMLLanguage{hu-HU}            % Sets the HTML meta language.
%\HTMLDescription{Latex tutorial}% Sets the HTML meta description.
\HTMLFirstPageTop{\fbox{AGOA}}
\HTMLPageTop{\fbox{AGOA}}
\HTMLPageBottom{@A Gondolkodás Öröme Alapítvány}
%\CSSFilename{lwarp_sagebrush.css}
\CSSFilename{tikz.css}
\renewcommand{\contentsname}{Tartalom}

\begin{document}
	
	\maketitle                      % Or titlepage/titlingpage environment.

	% An article abstract would go here.
	
	\tableofcontents                % MUST BE BEFORE THE FIRST SECTION BREAK!
	%\listoffigures

	
\chapter{TikZ gyorstalpaló}

\section{Alapok}

A \verb|\usepackage{tikzpicture}| kell a library implementálásához A \verb|\begin{tikzpicture}| és \verb|\end{tikzpicture}| parancsok közé kell helyezni a rajzolandó ábrát. A TikZ úgy működik, mint egy rajztábla. Egyesével kell az objektumokat rárajzolni, esetenként egy ciklusban többet is lehet egyszerre (lásd lejjebb). \textbf{Minden parancsot egy  ;-vel kell lezárni.}

A \verb|\begin{tikzpicture}["paraméterek"]| ebben a szögletes zárójelben kell megadni a rajztábla paramétereit. Ilyenek:
\begin{itemize}
    \item "\verb|scale = 3|"  -- a képet nyújtja, kivéve a betű méretet
    \item "\verb|xscale = 4, yscale = 5|"  -- ugyanez, csak merőlegesen affin képet ad
\end{itemize}

A rajzolásra két különböző, de általában mindenre elég parancs a \verb|\draw| és \verb|\filldraw| . A sima rajzolás csak körvonalat rajzol, a másik pedig automatikusan ugyanazzal a színnel kitölti az alakzatot. Mindkettő parancsnak meg kell mondani, hogy:

\begin{itemize}
    \item Hova: \verb|(x, y)|, \verb|(fok:hossz)|
    \item Mit: \verb|node|, \verb|--| (edge), \verb|circle|, \verb|rectangle|, \verb|arc|
    \item Stílusban: \verb|[color, ultra thin, fill]| -- ez lehet üres, ilyenkor a rajztábla stílusát használja
\end{itemize}

A node-ok kicsit trükkösebbek, róluk a gráfok részben lehet részletesebben olvasni.

\subsubsection{Kód}
\begin{SideBySideExample}
\begin{tikzpicture}[scale=3]
    %a köröknek a kp.-át és sugarát kell megadni
    \draw (0,0) circle (0.4 cm) [color = blue!90];
    \filldraw (1,0) circle (0.4 cm) [color = red!90];
    
    %a téglalapoknak a balalsó és jobbfelső csúcsait kell megadni
    \draw (2-0.4, -.4) rectangle (2+0.4, .4) [ultra thick, fill=black!20];
    
    %a törött vonalakat csúcsról csúcsra kell megadni
    \draw  (3-0.3, -0.3) -- (3-0.3, 0.4) -- (3+0.4, -0.4) -- (3+0.4, 0.4);
    
    %ami sokkal menőbb, például egy rácsbejáráshoz az íveltvonalak
    \draw[thick,rounded corners=8pt, color=pink!200] (4-0.3, -0.3) -- (4-0.3, 0.4) 
    -- (4+0.4, -0.4) -- (4+0.4, 0.4);
    
    %Ha a törött vonalat lezárnád érdemes a --cycle befejezést írni a kezdő csúcs 
    %megismétlése helyett.
\end{tikzpicture}
\end{SideBySideExample}

\subsubsection{Példa}
\begin{tikzpicture}[scale=3]
    %a köröknek a kp.-át és sugarát kell megadni
    \draw (0,0) circle (0.4 cm) [color = blue!90];
    \filldraw (1,0) circle (0.4 cm) [color = red!90];
    
    %a téglalapoknak a balalsó és jobbfelső csúcsait kell megadni
    \draw (2-0.4, -.4) rectangle (2+0.4, .4) [ultra thick, fill=black!20];
    
    %a törött vonalakat csúcsról csúcsra kell megadni
    \draw  (3-0.3, -0.3) -- (3-0.3, 0.4) -- (3+0.4, -0.4) -- (3+0.4, 0.4);
    
    %ami sokkal menőbb, például egy rácsbejáráshoz az íveltvonalak
    \draw[thick,rounded corners=8pt, color=pink!200] (4-0.3, -0.3) -- (4-0.3, 0.4) 
    -- (4+0.4, -0.4) -- (4+0.4, 0.4);
    
    %Ha a törött vonalat lezárnád érdemes a --cycle befejezést írni a kezdő csúcs %megismétlése helyett.
\end{tikzpicture}


\subsection{Illesztés}

Az első fejezetben leírtakat érdemes alkalmazni. A \verb|\clip| parancsot érdemes használni. Nem csak arra jó, hogy kivágjuk a kép egy részét, de beállítja a kép keretét, ha azzal kezdjük. Erre persze lehet használni a \verb|\useasboundingbox| parancsot amivel megadhatunk például egy téglalappal határolt fix keretét a képnek. Amit ezen kívül rajzoltál nem fogja megjeleníteni.

\subsubsection{Kód}
\selectlanguage{magyar}
\begin{Verbatim}
\begin{tikzpicture}[scale=3]        
    \draw (0,0) circle (0.4 cm) [color = blue!90];
    %Itt vágunk ami azt okozza, hogy az előző kör nem sérült
    \clip (-0.3, -0.3) rectangle (5, 0.3);
    \filldraw (1,0) circle (0.4 cm) [color = red!90];
    \draw (2-0.4, -.4) rectangle (2+0.4, .4) [ultra thick, fill=black!20];
    %Lehet relatív megadni a távolságokat, hogy ne kelljen mindent papíron kiszámolni
    %Ha csak sima +-t használsz, akkor a kezdő csúcstól viszonyít
    \draw  (3-0.3, -0.3) -- ++(0, 0.7) -- ++(0.7, -0.8) -- ++(0, 0.8);
    \draw[thick,rounded corners=8pt, color=pink!200]     (4-0.3, -0.3) -- (4-0.3, 0.4) -- (4+0.4, -0.4) -- (4+0.4, 0.4);
\end{tikzpicture}
\end{Verbatim}

\subsubsection{Példa}
\begin{tikzpicture}[scale=3]        
    \draw (0,0) circle (0.4 cm) [color = blue!90];
    %Itt vágunk ami azt okozza, hogy az előző kör nem sérült
    \clip (-0.3, -0.3) rectangle (5, 0.3);
    \filldraw (1,0) circle (0.4 cm) [color = red!90];
    \draw (2-0.4, -.4) rectangle (2+0.4, .4) [ultra thick, fill=black!20];
    %Lehet relatív megadni a távolságokat, hogy ne kelljen mindent papíron kiszámolni
    %Ha csak sima +-t használsz, akkor a kezdő csúcstól viszonyít
    \draw  (3-0.3, -0.3) -- ++(0, 0.7) -- ++(0.7, -0.8) -- ++(0, 0.8);
    \draw[thick,rounded corners=8pt, color=pink!200]     (4-0.3, -0.3) -- (4-0.3, 0.4) -- (4+0.4, -0.4) -- (4+0.4, 0.4);
\end{tikzpicture}

\subsection{Színek, egyebek}

Be lehet állítani vonalvastagságot, színt és még színátmenetes ábrát is egyszerű csinálni.
\begin{itemize}
    \item Vastagságok: \{\verb|ultra|, \verb|very|, \} + \{\verb|thin|, \verb|thick|\}
    \item Színek: \{ \verb|red|, \verb|green|, \verb|blue|, \verb|cyan|, \verb|magenta|, \verb|yellow|, \verb|black|, \verb|gray|, \verb|darkgray|, \verb|lightgray|, \verb|brown|, \verb|lime|, \verb|olive|, \verb|orange|, \verb|pink|, \verb|purple|, \verb|teal|, \verb|violet|, \verb|white| \}
    \item Vonal típusok: \{\verb|dashed|, \verb|dotted|\}
    \item Vonal összekötési lehetőségek (advanced): \begin{itemize}
        \item \verb|line cap = {round, rect, butt}|
        \item \verb|rounded corners = 5mm|
        \item \verb|line join = {round, bevel, mitern}|
        \end{itemize}
\end{itemize}

\subsubsection{Kód}
\selectlanguage{magyar}
\begin{Verbatim}
\begin{tikzpicture}[scale=3]
    \draw (0,0) circle (0.4) [color = green!70, fill = green!15, ultra thick];
    \draw (1,0) circle (0.4)     [color = green!70!black, fill = green!15, thick, dashed];
    \shade (2,0) circle (0.4) [top color = green];
    \shade (3,0) circle (0.4) [top color = green, bottom color = yellow];
    \shade (4,0) circle (0.4) [left color = green, right color = yellow];
\end{tikzpicture}
\end{Verbatim}

\begin{tikzpicture}[scale=3]
    \draw (0,0) circle (0.4) [color = green!70, fill = green!15, ultra thick];
    \draw (1,0) circle (0.4)     [color = green!70!black, fill = green!15, thick, dashed];
    \shade (2,0) circle (0.4) [top color = green];
    \shade (3,0) circle (0.4) [top color = green, bottom color = yellow];
    \shade (4,0) circle (0.4) [left color = green, right color = yellow];
\end{tikzpicture}

\section{Sokszögek rajzolása, for ciklusok}

Az, hogy lehet for ciklusokat írni, nagyban segít a valamilyen szempontból szimmetrikus ábrák elkészítésében. A for ciklusok hasonlóan más nyelvekhez bevezetnek egy változót, ami végig fut adott értékeken és végrehajtja a megadott parancsokat egyesével (jobb ha nem számít a sorrend).  Lehet egymásba ágyazott ciklusokat írni, de lehet párhuzamosan két vagy több változót egyszerre változtatni. Például \verb|\foreach \x in {1,2,3,4}{<commands>}| Ennél lehet komolyabb dolgokat is csinálni, lásd a példákat.

Eddig nem volt róla szó, de a hagyományos koordinátázás helyett lehet polárkoordinátákat is használni. \verb|(90:1cm)| -- 90 fok, 1 cm messze

A képet lehet transzformálni erre pár példa: \verb|xshift|, \verb|yshift|, \verb|rotate|

\subsubsection{Kód}
\selectlanguage{magyar}
\begin{Verbatim}
\begin{tikzpicture}[scale = 2, ultra thick]
    \foreach \n in {3, ..., 8}{    \draw (\n-3,0) \foreach \d in {1, ..., \n}{ %MAGIC DANGER        
            +(\d*360/\n:0.3cm) -- +(\d*360/\n + 360/\n:0.3cm)    }; 
            %Az, hogy ilyet lehet csinálni szerintem egyszerre undorító és hasznos
            %Ez kell ahhoz, hogy a szín mögé lehessen írni változót (nem igazán lehet képletet)
            \pgfmathsetmacro\i{\n*15-30} 
            \filldraw [xshift = \n-3, color = green!\i] (\n-3,-1) circle (0.3cm);
    }
\end{tikzpicture}
\end{Verbatim}

\subsubsection{Példa}
\begin{tikzpicture}[scale = 2, ultra thick]
    \foreach \n in {3, ..., 8}{    \draw (\n-3,0) \foreach \d in {1, ..., \n}{ %MAGIC DANGER        
            +(\d*360/\n:0.3cm) -- +(\d*360/\n + 360/\n:0.3cm)    }; 
            %Az, hogy ilyet lehet csinálni szerintem egyszerre undorító és hasznos
            %Ez kell ahhoz, hogy a szín mögé lehessen írni változót (nem igazán lehet képletet)
            \pgfmathsetmacro\i{\n*15-30} 
            \filldraw [xshift = \n-3, color = green!\i] (\n-3,-1) circle (0.3cm);
        }
\end{tikzpicture}

\section{Rácsok, szöveg beillesztése}

A \verb|\draw grid| parancsot lehet négyzetrács készítésre használni a \verb|\foreach| helyett. Meg kell adni a lépésközt és egy téglalapot ami határolja.

Szöveget beilleszteni úgy kell, hogy egy Node-ot töltünk fel szöveggel. Paraméterként meg lehet adni, hogy az adott pozícióhoz képest, hol helyezkedjen el a csúcs és így a szöveg, ezt az \verb|anchor=<direction>| paraméterrel lehet megadni. A \verb|fill=white| paraméter megadásával az is elérhető, hogy a szöveg/szám alatt megszakadjanak a vonalak, így egy sokkal esztétikusabb végeredményt kapunk. 

	
	%\appendix
	\chapter{Forr\'as}

\begin{enumerate}
    \item[] Forrás: \href{https://github.com/a-gondolkodas-orome/latex-tutorial}{github} 
    \item[] Honlap: \code{./pages/mainpage.tex}
    \item[] Fordítás: \code{python ./build\_html.py} 
    \item[] Generált oldal: \code{./docs/index.html} (\href{https://a-gondolkodas-orome.github.io/latex-tutorial/index.html}{online})
\end{enumerate}
        
A címekben egyes ékezet helyett a  '<karakter>-t kell használni. 

Egy weboldal megfelel egy \code{\chapter}-nek a mostani beállítás (\code{FileDepth}) szerint. 
Új fájl esetén \code{\chapter\{title\_of\_chapter\}} paranccsal kezdődjön a fájl. A \code{mainpage.tex} fájlban szükséges az \code{\include\{name\_of\_file\}}.


	
%	\ForceHTMLPage      % HTML index will be on its own page.
%	\ForceHTMLTOC       % HTML index will have its own toc entry.
	\printindex
\end{document}
