
% Save this as tutorial.tex for the lwarp package tutorial.

\documentclass[12pt,a4paper]{book}

\usepackage{iftex}

% --- LOAD FONT SELECTION AND ENCODING BEFORE LOADING LWARP ---

\ifPDFTeX
\usepackage{lmodern}            % pdflatex or dvi latex
\usepackage[utf8]{inputenc}
\else
\usepackage{fontspec}           % XeLaTeX or LuaLaTeX
\fi
\usepackage[magyar]{babel}

% --- LWARP IS LOADED NEXT ---
\usepackage[
% Filename of the homepage.
   HomeHTMLFilename=index,     
   HTMLFilename={node-},       % Filename prefix of other pages.
%   IndexLanguage=english,      % Language for xindy index, glossary.
%   REMARK (szgabbor): After commenting, I was able to compile
%   xindex,
%   xindy,
   %latexmk,                    % Use latexmk to compile.
%   OSWindows,                  % Force Windows. (Usually automatic.)
%   mathjax,                    % Use MathJax to display math.
]{lwarp}
% \boolfalse{FileSectionNames}  % If false, numbers the files.

% --- LOAD PDFLATEX MATH FONTS HERE ---
% PREAMBULUM IDE

\usepackage{amssymb}
\usepackage{amsmath}
\usepackage{graphicx}
\usepackage{tikz}
\usepackage{fancyvrb-ex}
\usepackage[a4paper]{geometry}

% --- OTHER PACKAGES ARE LOADED AFTER LWARP ---
%\usepackage{makeidx} \makeindex
\usepackage{xindex} \makeindex
%\usepackage{xindex} %\makeindex
%\usepackage[abbreviations,symbols,xindy]{glossaries-extra} 
\makeindex
\usepackage{xcolor}             % (Demonstration purposes only.)

\usepackage{mdframed} 
\definecolor{bgInline}{rgb}{0.9, 0.9, 0.9}
\definecolor{bgBox}{rgb}{0.8, 0.8, 0.8}

\mdfdefinestyle{box}{
    linecolor=bgBox,
	backgroundcolor=bgBox,
	hidealllines=true,
	roundcorner=10pt,
}

\usepackage{minted} % http://tug.ctan.org/macros/latex/contrib/minted/minted.pdf

% https://tex.stackexchange.com/questions/504036/using-mdframed-with-inputminted
\let\inputmintedorg\inputminted
\newcommand{\mintedframe}[3][]{%
\begin{mdframed}[style=box]
\inputminted[#1]{#2}{#3}
\end{mdframed}
}

% https://tex.stackexchange.com/questions/89873/repeating-environment-contents-twice
\newenvironment{tikzExample}
 {\VerbatimOut{\jobname.tmp}}
 {\endVerbatimOut
  \mintedframe[tabsize=4,breaklines,breaksymbolleft={}]{latex}{\jobname.tmp}
  \input{\jobname.tmp}}

\newcommand{\code}[1]{\mintinline[bgcolor=bgInline]{latex}{#1}}

\usepackage{hyperref,cleveref}  % LOAD THESE LAST!

% --- LATEX AND HTML CUSTOMIZATION ---
\title{TikZ gyorstalpaló, példák}
\author{Bertalan Dávid, Alexy Marcell (technikai megvalósítás)}
\setcounter{tocdepth}{1}        % Include subsections in the \TOC.
\setcounter{secnumdepth}{2}     % Number down to subsections.
\setcounter{FileDepth}{0}       % 1: Split \HTML\ files at sections ; 0: split at chapter
\booltrue{CombineHigherDepths}  % Combine parts/chapters/sections
\setcounter{SideTOCDepth}{1}    % Include subsections in the side\TOC
%\HTMLTitle{Webpage Title}       % Overrides \title for the web page.
%\HTMLAuthor{\theauthor}        % Sets the HTML meta author tag.
\HTMLLanguage{hu-HU}            % Sets the HTML meta language.
%\HTMLDescription{Latex tutorial}% Sets the HTML meta description.
\HTMLFirstPageTop{\fbox{AGOA}}
\HTMLPageTop{\fbox{AGOA}}
\HTMLPageBottom{@A Gondolkodás Öröme Alapítvány}
%\CSSFilename{lwarp_sagebrush.css}
\CSSFilename{tikz.css}
\renewcommand{\contentsname}{Tartalom}

\begin{document}
	
	\maketitle                      % Or titlepage/titlingpage environment.

	% An article abstract would go here.
	
	\tableofcontents                % MUST BE BEFORE THE FIRST SECTION BREAK!
	%\listoffigures

	\chapter{TikZ gyorstalpal\'o}

\section{Alapok}

A \code{\usepackage{tikzpicture}} kell a library implementálásához A \code{\begin{tikzpicture}} és \code{\end{tikzpicture}} parancsok közé kell helyezni a rajzolandó ábrát. A TikZ úgy működik, mint egy rajztábla. Egyesével kell az objektumokat rárajzolni, esetenként egy ciklusban többet is lehet egyszerre (lásd lejjebb). \textbf{Minden parancsot egy  ;-vel kell lezárni.}

A \code{\begin{tikzpicture}["paraméterek"]} ebben a szögletes zárójelben kell megadni a rajztábla paramétereit. Ilyenek:
\begin{itemize}
    \item \code{scale = 3}  -- a képet nyújtja, kivéve a betű méretet
    \item \code{xscale = 4, yscale = 5}  -- ugyanez, csak merőlegesen affin képet ad
\end{itemize}

A rajzolásra két különböző, de általában mindenre elég parancs a \code{\draw} és \code{\filldraw} . A sima rajzolás csak körvonalat rajzol, a másik pedig automatikusan ugyanazzal a színnel kitölti az alakzatot. Mindkettő parancsnak meg kell mondani, hogy:

\begin{itemize}
    \item Hova: \code{(x, y)}, \code{(fok:hossz)}
    \item Mit: \code{node}, \code{--} (edge), \code{circle}, \code{rectangle}, \code{arc}
    \item Stílusban: \code{[color, ultra thin, fill]} -- ez lehet üres, ilyenkor a rajztábla stílusát használja
\end{itemize}

A node-ok kicsit trükkösebbek, róluk a gráfok részben lehet részletesebben olvasni.

\begin{tikzExample}
\begin{tikzpicture}[scale=3]
    %a köröknek a kp.-át és sugarát kell megadni
    \draw (0,0) circle (0.4 cm) [color = blue!90];
    \filldraw (1,0) circle (0.4 cm) [color = red!90];
    
    %a téglalapoknak a balalsó és jobbfelső csúcsait kell megadni
    \draw (2-0.4, -.4) rectangle (2+0.4, .4) [ultra thick, fill=black!20];
    
    %a törött vonalakat csúcsról csúcsra kell megadni
    \draw  (3-0.3, -0.3) -- (3-0.3, 0.4) -- (3+0.4, -0.4) -- (3+0.4, 0.4);
    
    %ami sokkal menőbb, például egy rácsbejáráshoz az íveltvonalak
    \draw[thick,rounded corners=8pt, color=pink!200] (4-0.3, -0.3) -- (4-0.3, 0.4) 
    -- (4+0.4, -0.4) -- (4+0.4, 0.4);
    
    %Ha a törött vonalat lezárnád érdemes a --cycle befejezést írni a kezdő csúcs 
    %megismétlése helyett.
\end{tikzpicture}
\end{tikzExample}


\subsection{Illesztés}

Az első fejezetben leírtakat érdemes alkalmazni. A \code{\clip} parancsot érdemes használni. Nem csak arra jó, hogy kivágjuk a kép egy részét, de beállítja a kép keretét, ha azzal kezdjük. Erre persze lehet használni a \code{\useasboundingbox} parancsot amivel megadhatunk például egy téglalappal határolt fix keretét a képnek. Amit ezen kívül rajzoltál nem fogja megjeleníteni.

\begin{tikzExample}
\begin{tikzpicture}[scale=3]        
    \draw (0,0) circle (0.4 cm) [color = blue!90];
    %Itt vágunk ami azt okozza, hogy az előző kör nem sérült
    \clip (-0.3, -0.3) rectangle (5, 0.3);
    \filldraw (1,0) circle (0.4 cm) [color = red!90];
    \draw (2-0.4, -.4) rectangle (2+0.4, .4) [ultra thick, fill=black!20];
    %Lehet relatív megadni a távolságokat, hogy ne kelljen mindent papíron kiszámolni
    %Ha csak sima +-t használsz, akkor a kezdő csúcstól viszonyít
    \draw  (3-0.3, -0.3) -- ++(0, 0.7) -- ++(0.7, -0.8) -- ++(0, 0.8);
    \draw[thick,rounded corners=8pt, color=pink!200]     (4-0.3, -0.3) -- (4-0.3, 0.4) -- (4+0.4, -0.4) -- (4+0.4, 0.4);
\end{tikzpicture}
\end{tikzExample}

\subsection{Színek, egyebek}

Be lehet állítani vonalvastagságot, színt és még színátmenetes ábrát is egyszerű csinálni.
\begin{itemize}
    \item Vastagságok: \{\code{ultra}, \code{very}, \} + \{\code{thin}, \code{thick}\}
    \item Színek: \{ \code{red}, \code{green}, \code{blue}, \code{cyan}, \code{magenta}, \code{yellow}, \code{black}, \code{gray}, \code{darkgray}, \code{lightgray}, \code{brown}, \code{lime}, \code{olive}, \code{orange}, \code{pink}, \code{purple}, \code{teal}, \code{violet}, \code{white} \}
    \item Vonal típusok: \{\code{dashed}, \code{dotted}\}
    \item Vonal összekötési lehetőségek (advanced): \begin{itemize}
        \item \code{line cap = {round, rect, butt}}
        \item \code{rounded corners = 5mm}
        \item \code{line join = {round, bevel, mitern}}
        \end{itemize}
\end{itemize}

\begin{tikzExample}
\begin{tikzpicture}[scale=3]
    \draw (0,0) circle (0.4) [color = green!70, fill = green!15, ultra thick];
    \draw (1,0) circle (0.4)     [color = green!70!black, fill = green!15, thick, dashed];
    \shade (2,0) circle (0.4) [top color = green];
    \shade (3,0) circle (0.4) [top color = green, bottom color = yellow];
    \shade (4,0) circle (0.4) [left color = green, right color = yellow];
\end{tikzpicture}
\end{tikzExample}

\section{Sokszögek rajzolása, for ciklusok}

Az, hogy lehet for ciklusokat írni, nagyban segít a valamilyen szempontból szimmetrikus ábrák elkészítésében. A for ciklusok hasonlóan más nyelvekhez bevezetnek egy változót, ami végig fut adott értékeken és végrehajtja a megadott parancsokat egyesével (jobb ha nem számít a sorrend).  Lehet egymásba ágyazott ciklusokat írni, de lehet párhuzamosan két vagy több változót egyszerre változtatni. Például \code{\foreach \x in {1,2,3,4}{<commands>}} Ennél lehet komolyabb dolgokat is csinálni, lásd a példákat.

Eddig nem volt róla szó, de a hagyományos koordinátázás helyett lehet polárkoordinátákat is használni. \code{(90:1cm)} -- 90 fok, 1 cm messze

A képet lehet transzformálni erre pár példa: \code{xshift}, \code{yshift}, \code{rotate}

\begin{tikzExample}
\begin{tikzpicture}[scale = 2, ultra thick]
    \foreach \n in {3, ..., 8}{    \draw (\n-3,0) \foreach \d in {1, ..., \n}{ %MAGIC DANGER        
            +(\d*360/\n:0.3cm) -- +(\d*360/\n + 360/\n:0.3cm)    }; 
            %Az, hogy ilyet lehet csinálni szerintem egyszerre undorító és hasznos
            %Ez kell ahhoz, hogy a szín mögé lehessen írni változót (nem igazán lehet képletet)
            \pgfmathsetmacro\i{\n*15-30} 
            \filldraw [xshift = \n-3, color = green!\i] (\n-3,-1) circle (0.3cm);
    }
\end{tikzpicture}
\end{tikzExample}

\section{Rácsok, szöveg beillesztése}

A \code{\draw grid} parancsot lehet négyzetrács készítésre használni a \code{\foreach} helyett. Meg kell adni a lépésközt és egy téglalapot ami határolja.

Szöveget beilleszteni úgy kell, hogy egy Node-ot töltünk fel szöveggel. Paraméterként meg lehet adni, hogy az adott pozícióhoz képest, hol helyezkedjen el a csúcs és így a szöveg, ezt az \code{anchor=<direction>} paraméterrel lehet megadni. A \code{fill=white} paraméter megadásával az is elérhető, hogy a szöveg/szám alatt megszakadjanak a vonalak, így egy sokkal esztétikusabb végeredményt kapunk. 

Itt különösen kiemelném a \code{\clip} parancs fontosságát. Ha egy ábrát szeretnék nagyban és kicsiben is használni elég megismételni a kódot és megadunk egy keretet, ahol kíváncsiak vagyunk az ábra részleteire. 

\begin{tikzExample}
\begin{tikzpicture}[scale = 3]
	\clip (-1.2, -0.8) rectangle (4.2,2.2); %Ez csak azért, hogy jobban ráférjen a honlapra
	%grid
	\draw[step = 0.5, color=gray] (-2.1,-2.1) grid (2.1,2.1);
	%axes
	\draw[->, ultra thick] (0,-2.2) -- (0,2.2);
	\draw[->, ultra thick] (-2.2,0) -- (2.2,0);
	%texts
	\draw (0,0) [fill = white, anchor = north east] node {\large $O$};
	
	%y-tengely
	\foreach \label in {1, 2, 3, 4}
	\pgfmathsetmacro\pos{\label/2}
	\draw [ultra thick](-1pt,\pos) -- (1pt, \pos) node [fill = white, left, xshift = -7pt] {$\label$};
	\foreach \label in {-1, -2, -3, -4}
	\pgfmathsetmacro\pos{\label/2}
	\draw [ultra thick](-1pt,\pos) -- (1pt, \pos) node [fill = white, left, xshift = -7pt] {$\label$};
	
	%x-tengely		
	\foreach \label in {1, 2, 3, 4}
	\pgfmathsetmacro\pos{\label/2}
	\draw [ultra thick](\pos, 1pt) -- (\pos, -1pt) node [fill = white, below, yshift = -2pt] {$\label$};
	\foreach \label in {-1, -2, -3, -4}
	\pgfmathsetmacro\pos{\label/2}
	\draw [ultra thick](\pos, 1pt) -- (\pos, -1pt) node [fill = white, below, yshift = -2pt, xshift = -3pt] {$\label$};
	
	%ábra
	\draw (1, 0.5) node [color=red, anchor = south west] {$A$};
	\draw (0.5, 1.5) node [color=blue, anchor = south west] {$B$};
	\draw (0.5,1.5) node [color=blue, circle, fill=blue, scale =0.7] {};
	\draw [->, green, dashed, ultra thick, opacity=0.5] (1, 0.5) -- (0.5, 1.5);
	\draw (1, 0.5) node [color=red, circle, fill=red, scale =0.7] {};
	\draw[xshift=2.1cm, yshift=1cm] node[right,text width=5cm]
	{Az ábrán látható {\color{red} $A$} pontból megy a {\color{blue} $B$} pontba egy {\color{green} vektor}.};
\end{tikzpicture}
\end{tikzExample}

\section{Gráfok}

Lehet gráfokat úgy definiálni, hogy a csúcsokat megadjuk és utána az élek már a meglévő objektumainkat (csúcsok) kössék össze. Ez azért hasznos, mert rugalmasabb lesz az ábra. Ha esetleg változtatnánk a gráfon egy új csúcs behozásával nem kell az egész ábrát koordinátánként átírni. Elég csak a csúcsokat áthelyezni, a többit a TikZ megcsinálja nekünk. Ami még különösen hasznos, hogy tudunk a programban a csúcsoknak nevet adni és utána ezt a nevet használni referenciaként, hogy egy sokkal átláthatóbb kódot kapjunk végeredményül. Ez nem összekeverendő a csúcshoz tartozó szöveggel.

Amit szintén itt mutatnék be az a dinamikus stílus kezelés. Lehet ugyanis általunk előre definiált stílusokat megadni, hogy utána csak elég legyen annyit írni, hogy \code{[fontos]} vagy \code{[seged]}. Ezzel is azt érjük el, hogy olvashatóbb és egységesen változtathatóbb lesz a kód és így az ábránk.

A csúcsok és élek szövegezésére is sok lehetőséget ad a TikZ. A \code{label=<direction>:<text>} paraméter, akár többszöri használatával tudunk mindenféle szöveggel/névvel ellátni az ábránkat.

Lehet az éleket hajlítani, kígyósítani és egyéb stilisztikai trükköket alkalmazni. Erre azt ajánlom, hogy a dokumentációt érdemes olvasgatni. A következő részben írok a görbe vonalakról, ott érdemes erről olvasni.

\begin{tikzExample}
\usetikzlibrary{positioning,backgrounds}
\begin{tikzpicture}[auto, node distance = 1cm and 2cm]
	\tikzstyle{StartEnd}=[rectangle,draw=blue!50, fill=blue!20,thick, 			inner sep=0pt,minimum size=6mm]
	\tikzstyle{alayer}=[circle,draw=red!80,fill=red!20,thick, inner sep=0pt,minimum size=6mm]
	\tikzstyle{blayer}=[circle,draw=red!80,fill=red!40,thick, inner sep=0pt,minimum size=6mm]
	\tikzstyle{se-edge}=[->,very thick, color=blue!30]
	\tikzstyle{in-edge}=[->,very thick, color=red!30]
	
	%Nodes
	\node[StartEnd] (Start) [label = 135:\color{blue}\Large$S$] {};
	
	\node[alayer] (a3) [right = of Start, label=above:$a3$] {};
	\node[alayer] (a2) [above = of a3, label=above:$a2$] {};
	\node[alayer] (a1) [above = of a2, label=above:$a1$] {};
	\node[alayer] (a4) [below = of a3, label=above:$a4$] {};
	\node[alayer] (a5) [below = of a4, label=above:$a5$] {};
	
	\node[blayer] (b3) [right = of a3, label=above:$b3$] {};
	\node[blayer] (b2) [above = of b3, label=above:$b2$] {};
	\node[blayer] (b1) [above = of b2, label=above:$b1$] {};
	\node[blayer] (b4) [below = of b3, label=above:$b4$] {};
	\node[blayer] (b5) [below = of b4, label=above:$b5$] {};
	
	\node[StartEnd] (End)[right = of b3,label=45:\color{blue}\Large$C$] {};
	
	%Edges
	\draw[se-edge] (Start) to [out=45, in=180] (a1);
	\draw[se-edge] (Start) to [out=22.5, in=180] (a2);
	\draw[se-edge] (Start) to [out=0, in=180] (a3);
	\draw[se-edge] (Start) to [out=360-22.5, in=180] (a4);
	\draw[se-edge] (Start) to [out=360-45, in=180] (a5);
	
	\draw[se-edge] (b1) to [out=0, in=180-45] (End);
	\draw[se-edge] (b2) to [out=0, in=180-22.5] (End);
	\draw[se-edge] (b3) to [out=0, in=180] (End);
	\draw[se-edge] (b4) to [out=0, in=180+22.5] (End);
	\draw[se-edge] (b5) to [out=0, in=180+45] (End);
	
	\draw[in-edge] (a1) to (b2);
	\draw[in-edge] (a2) to (b1);
	\draw[in-edge] (a2) to (b5);
	\draw[in-edge] (a3) to (b5);
	\draw[in-edge] (a4) to (b1);
	\draw[in-edge] (a5) to (b3);
	\draw[in-edge] (a5) to (b5);
	
	%Layers
	\begin{pgfonlayer}{background}
		\filldraw [fill=black!20, draw=black] (a5.south -| a5.west) rectangle (a1.north -| a1.east);
		\filldraw [fill=black!20, draw=black] (b5.south -| b5.west) rectangle (b1.north -| b1.east);
	\end{pgfonlayer}
	
\end{tikzpicture}
\end{tikzExample}


	
	%\appendix
	\chapter{Forr\'as}

\begin{enumerate}
    \item[-] \href{https://github.com/a-gondolkodas-orome/latex-tutorial}{Github link} 
    \item[-] main.tex: \code{./pages/mainpage.tex}
    \item[-] Fordítás: \code{python ./build_html.py} 
    \item[-] index.html: \code{./docs/index.html} 
    \item[-] \href{https://a-gondolkodas-orome.github.io/latex-tutorial/index.html}{Honlap}
    \item[-] \href{https://a-gondolkodas-orome.github.io/latex-tutorial/mainpage.pdf}{PDF}
\end{enumerate}
        
Egy weboldal megfelel egy \code{\chapter}-nek a mostani beállítás szerint (ld. FileDepth). Fontos, hogy új fájl esetén \code{\chapter{title_of_chapter}} paranccsal kezdődjön a fájl. A \code{mainpage.tex} fájlban szükséges az \code{\include{name_of_file}}.


	
%	\ForceHTMLPage      % HTML index will be on its own page.
%	\ForceHTMLTOC       % HTML index will have its own toc entry.
	\printindex
\end{document}
